\documentclass[12pt,a4paper]{article}
\usepackage[utf8]{inputenc}
\usepackage[french]{babel}
\usepackage{amsmath}
\usepackage{amssymb}
\usepackage{physics}
\usepackage{siunitx}
\usepackage{geometry}
\geometry{margin=2.5cm}

\title{Calculs mathématiques pour les constellations Walker Delta}
\author{}
\date{}

\begin{document}

\maketitle

\section{Introduction}

Les constellations Walker Delta sont définies par une notation compacte $T/P/F$ où :
\begin{itemize}
    \item $T$ : nombre total de satellites
    \item $P$ : nombre de plans orbitaux
    \item $F$ : facteur de phase (avec $0 \leq F < P$)
\end{itemize}

Cette section présente l'ensemble des calculs permettant de modéliser le mouvement des satellites dans ces constellations.

\section{Hypothèses simplificatrices}

Pour permettre une analyse mathématique tractable et une simulation efficace, les hypothèses suivantes sont adoptées :

\subsection{Hypothèses orbitales}

\subsubsection{Orbites circulaires}

Les trajectoires des satellites sont considérées comme \textbf{parfaitement circulaires}. Dans la réalité, toute orbite possède une excentricité non nulle, mais pour les constellations LEO bien maintenues :

\begin{equation}
    e = 0 \quad \Rightarrow \quad a = r = \text{constante}
\end{equation}

Cette hypothèse implique :
\begin{itemize}
    \item Vitesse orbitale constante : $v = \sqrt{\frac{GM}{r}}$
    \item Vitesse angulaire constante : $\omega = \sqrt{\frac{GM}{r^3}}$
    \item Période orbitale fixe et déterministe
\end{itemize}

En pratique, les manœuvres orbitales maintiennent l'excentricité $e < 10^{-3}$ pour les constellations opérationnelles.

\subsubsection{Problème à deux corps}

Le mouvement de chaque satellite est modélisé selon le \textbf{problème de Kepler à deux corps} (Terre-satellite), négligeant :

\begin{itemize}
    \item Les perturbations gravitationnelles de la Lune et du Soleil
    \item Les harmoniques du champ gravitationnel terrestre ($J_2, J_3, \ldots$)
    \item La pression de radiation solaire
    \item La traînée atmosphérique résiduelle
\end{itemize}

Cette approximation est valide pour des simulations à court terme (quelques périodes orbitales). Pour des simulations à long terme, ces perturbations doivent être prises en compte.

\subsection{Hypothèses géométriques}

\subsubsection{Terre sphérique}

La Terre est modélisée comme une \textbf{sphère parfaite} de rayon $R_{\oplus} = 6371$ km, correspondant au rayon moyen terrestre. Les effets de l'aplatissement aux pôles (ellipsoïde de référence WGS84) sont négligés :

\begin{equation}
    \text{Aplatissement : } f = \frac{a - b}{a} \approx \frac{1}{298.257} \approx 0.0034 \quad (\text{négligé})
\end{equation}

où $a$ et $b$ sont les demi-grands et demi-petits axes de l'ellipsoïde terrestre.

\subsubsection{Rotation terrestre uniforme}

La rotation de la Terre est considérée comme \textbf{uniforme} avec une vitesse angulaire constante :

\begin{equation}
    \omega_{\oplus} = \frac{2\pi}{T_{\text{sidéral}}} = \frac{2\pi}{86164.0905} \approx 7.292115 \times 10^{-5} \text{ rad/s}
\end{equation}

Les variations de la vitesse de rotation terrestre (variations du jour sidéral, nutation, précession) sont négligées.

\subsection{Hypothèses de propagation}

\subsubsection{Propagation en ligne droite}

Les signaux électromagnétiques entre satellites se propagent en \textbf{ligne droite} à la vitesse de la lumière $c$ dans le vide. Les effets de réfraction atmosphérique et ionosphérique sont négligés.

\subsubsection{Visibilité binaire}

La visibilité entre deux satellites est considérée comme un état \textbf{binaire} (visible/non visible) basé uniquement sur l'occultation terrestre :

\begin{equation}
    \text{Visible}(i,j) =
    \begin{cases}
        1 & \text{si } d_{\min}^{ij} > R_{\oplus} \\
        0 & \text{sinon}
    \end{cases}
\end{equation}

Les marges de Fresnel et les zones de diffraction sont négligées.

\subsection{Hypothèses dynamiques}

\subsubsection{Absence de manœuvres}

Les satellites maintiennent leur orbite sans effectuer de \textbf{manœuvres orbitales} durant la simulation. En réalité, des corrections périodiques sont nécessaires pour :
\begin{itemize}
    \item Compenser la traînée atmosphérique résiduelle
    \item Corriger les dérives dues aux perturbations gravitationnelles
    \item Maintenir le phasage relatif entre satellites
\end{itemize}

\subsubsection{Configuration stationnaire}

La topologie de la constellation (nombre de satellites, plans, phasage) est considérée comme \textbf{fixe}. Les scenarios de dégradation (défaillance d'un satellite) ou d'évolution (ajout de satellites) ne sont pas modélisés.

\subsection{Hypothèses temporelles}

\subsubsection{Synchronisation parfaite}

Tous les satellites sont considérés comme \textbf{parfaitement synchronisés} avec une horloge de référence commune. Les dérives d'horloge et les effets relativistes (dilatation du temps) sont négligés.

\subsubsection{État initial connu}

L'état orbital initial de chaque satellite (position et vitesse) est supposé \textbf{parfaitement connu} sans incertitude. Dans la pratique, les erreurs de détermination d'orbite peuvent atteindre quelques mètres à quelques dizaines de mètres.

\subsection{Domaine de validité}

Ces hypothèses sont valides dans le contexte suivant :

\begin{itemize}
    \item \textbf{Altitude} : $350 \leq h \leq 1200$ km (orbite basse terrestre - LEO)
    \item \textbf{Durée de simulation} : Quelques périodes orbitales ($\sim$ heures à jours)
    \item \textbf{Précision requise} : Analyse de topologie et de connectivité (non positionnement précis)
    \item \textbf{Objectif} : Évaluation de la stabilité des liens et des performances de routage
\end{itemize}

Pour des applications nécessitant une précision de positionnement métrique ou pour des simulations sur plusieurs mois, un modèle de propagation plus sophistiqué (SGP4, propagateur numérique avec perturbations) serait nécessaire.

\subsection{Justification des hypothèses}

Les simplifications adoptées permettent :

\begin{enumerate}
    \item \textbf{Calcul analytique} : Les formules en forme close permettent une implémentation efficace
    \item \textbf{Simulation temps réel} : La propagation simple permet d'accélérer la simulation (facteur 1000×)
    \item \textbf{Compréhension intuitive} : Les trajectoires circulaires facilitent l'analyse géométrique
    \item \textbf{Stabilité numérique} : Absence d'accumulation d'erreurs sur plusieurs révolutions
\end{enumerate}

L'erreur introduite par l'hypothèse d'orbite circulaire reste inférieure à 1\% pour une excentricité $e < 0.01$, ce qui est largement suffisant pour l'étude de la topologie des constellations.

\section{Paramètres orbitaux fondamentaux}

\subsection{Constantes physiques}

Les constantes suivantes sont utilisées pour les calculs :

\begin{align}
    R_{\oplus} &= 6371 \text{ km} \quad \text{(rayon terrestre)} \\
    GM &= 398600.4418 \text{ km}^3\text{s}^{-2} \quad \text{(paramètre gravitationnel standard)} \\
    c &= 299792.458 \text{ km/s} \quad \text{(vitesse de la lumière)} \\
    \omega_{\oplus} &= \frac{2\pi}{86400} \text{ rad/s} \quad \text{(vitesse de rotation terrestre)}
\end{align}

\subsection{Rayon orbital}

Pour une altitude $h$ donnée, le rayon orbital $r$ est :

\begin{equation}
    r = R_{\oplus} + h
\end{equation}

Pour les constellations LEO (Low Earth Orbit), typiquement : $350 \leq h \leq 1200$ km.

\subsection{Vitesse angulaire orbitale}

La vitesse angulaire $\omega$ d'un satellite sur son orbite circulaire est donnée par la troisième loi de Kepler :

\begin{equation}
    \omega = \sqrt{\frac{GM}{r^3}}
\end{equation}

Cette vitesse est exprimée en rad/s et représente la variation de l'anomalie vraie par unité de temps.

\subsection{Vitesse orbitale linéaire}

La vitesse linéaire $v$ du satellite sur son orbite est :

\begin{equation}
    v = \sqrt{\frac{GM}{r}} = \omega \cdot r
\end{equation}

\subsection{Période orbitale}

La période orbitale $T_{\text{orb}}$, temps nécessaire pour effectuer une révolution complète, est :

\begin{equation}
    T_{\text{orb}} = \frac{2\pi}{\omega} = 2\pi\sqrt{\frac{r^3}{GM}}
\end{equation}

Exprimée en minutes :

\begin{equation}
    T_{\text{orb}}[\text{min}] = \frac{1}{60} \cdot 2\pi\sqrt{\frac{r^3}{GM}}
\end{equation}

Pour une orbite LEO à 550 km d'altitude, $T_{\text{orb}} \approx 95.6$ minutes.

\section{Éléments orbitaux Keplériens}

Chaque satellite est caractérisé par ses éléments orbitaux. Pour les constellations Walker Delta avec orbites circulaires :

\begin{itemize}
    \item \textbf{Demi-grand axe} : $a = r$ (orbite circulaire)
    \item \textbf{Excentricité} : $e = 0$ (orbite circulaire)
    \item \textbf{Inclinaison} : $i$ (angle entre plan orbital et plan équatorial)
    \item \textbf{RAAN} : $\Omega$ (Right Ascension of Ascending Node)
    \item \textbf{Argument du périgée} : $\omega = 0$ (non défini pour orbite circulaire)
    \item \textbf{Anomalie vraie} : $\nu$ (position angulaire du satellite)
\end{itemize}

\section{Distribution Walker Delta}

\subsection{Calcul du RAAN pour chaque plan}

Les $P$ plans orbitaux sont distribués uniformément autour de l'équateur. Pour le plan $p$ (avec $p \in \{0, 1, \ldots, P-1\}$) :

\begin{equation}
    \Omega_p = \frac{360°}{P} \cdot p = \frac{2\pi}{P} \cdot p
\end{equation}

Cette distribution assure une couverture azimutale uniforme.

\subsection{Répartition des satellites par plan}

Le nombre de satellites par plan n'est pas nécessairement uniforme. Soit $N_p$ le nombre de satellites dans le plan $p$ :

\begin{align}
    N_p &= \left\lfloor \frac{T}{P} \right\rfloor + \begin{cases}
        1 & \text{si } p < (T \bmod P) \\
        0 & \text{sinon}
    \end{cases} \\
    \sum_{p=0}^{P-1} N_p &= T
\end{align}

Les plans avec indices inférieurs reçoivent un satellite supplémentaire si $T$ n'est pas divisible par $P$.

\subsection{Phasage Walker Delta}

Pour le satellite $s$ dans le plan $p$ (avec $s \in \{0, 1, \ldots, N_p-1\}$), l'anomalie vraie initiale $\nu_{p,s}$ est calculée selon le schéma de phasage Walker Delta :

\begin{equation}
    \nu_{p,s} = \frac{360°}{N_p} \cdot s + \frac{360° \cdot F}{T} \cdot p
\end{equation}

Ou en radians :

\begin{equation}
    \nu_{p,s} = \frac{2\pi}{N_p} \cdot s + \frac{2\pi \cdot F}{T} \cdot p
\end{equation}

Le terme $\frac{360°}{N_p} \cdot s$ assure une distribution uniforme des satellites dans chaque plan.

Le terme $\frac{360° \cdot F}{T} \cdot p$ introduit un décalage de phase entre plans adjacents, contrôlé par le facteur $F$.

\section{Position d'un satellite}

\subsection{Coordonnées dans le repère orbital}

Dans le référentiel du plan orbital (orbite circulaire), la position du satellite est :

\begin{align}
    x_{\text{orb}} &= r \cos(\nu) \\
    y_{\text{orb}} &= 0 \\
    z_{\text{orb}} &= r \sin(\nu)
\end{align}

\subsection{Transformation vers le repère inertiel}

La transformation du repère orbital vers le repère inertiel (ECI - Earth-Centered Inertial) implique deux rotations successives :

\begin{enumerate}
    \item Rotation d'angle $i$ (inclinaison) autour de l'axe des nœuds
    \item Rotation d'angle $\Omega$ (RAAN) autour de l'axe polaire
\end{enumerate}

La position dans le repère inertiel est donnée par :

\begin{equation}
    \begin{pmatrix}
        x \\
        y \\
        z
    \end{pmatrix}
    =
    \begin{pmatrix}
        \cos\Omega & -\sin\Omega & 0 \\
        \sin\Omega & \cos\Omega & 0 \\
        0 & 0 & 1
    \end{pmatrix}
    \begin{pmatrix}
        1 & 0 & 0 \\
        0 & \cos i & -\sin i \\
        0 & \sin i & \cos i
    \end{pmatrix}
    \begin{pmatrix}
        x_{\text{orb}} \\
        0 \\
        z_{\text{orb}}
    \end{pmatrix}
\end{equation}

Soit, en développant :

\begin{align}
    x &= x_{\text{orb}} \cos\Omega - z_{\text{orb}} \cos i \sin\Omega \\
    y &= z_{\text{orb}} \sin i \\
    z &= x_{\text{orb}} \sin\Omega + z_{\text{orb}} \cos i \cos\Omega
\end{align}

En substituant les expressions de $x_{\text{orb}}$ et $z_{\text{orb}}$ :

\begin{align}
    x &= r\cos\nu \cos\Omega - r\sin\nu \cos i \sin\Omega \\
    y &= r\sin\nu \sin i \\
    z &= r\cos\nu \sin\Omega + r\sin\nu \cos i \cos\Omega
\end{align}

\section{Évolution temporelle}

\subsection{Propagation de l'anomalie vraie}

Pour une orbite circulaire, l'anomalie vraie évolue linéairement avec le temps :

\begin{equation}
    \nu(t) = \nu_0 + \omega \cdot t
\end{equation}

où $\nu_0$ est l'anomalie vraie initiale (calculée selon le phasage Walker Delta) et $t$ est le temps écoulé depuis l'epoch.

\subsection{Position en fonction du temps}

En combinant les équations précédentes, la position complète du satellite à l'instant $t$ est :

\begin{align}
    x(t) &= r[\cos(\nu_0 + \omega t) \cos\Omega - \sin(\nu_0 + \omega t) \cos i \sin\Omega] \\
    y(t) &= r\sin(\nu_0 + \omega t) \sin i \\
    z(t) &= r[\cos(\nu_0 + \omega t) \sin\Omega + \sin(\nu_0 + \omega t) \cos i \cos\Omega]
\end{align}

\section{Rotation terrestre et coordonnées fixes}

\subsection{Repère ECEF}

Le repère ECEF (Earth-Centered Earth-Fixed) tourne avec la Terre. L'angle de rotation terrestre à l'instant $t$ est :

\begin{equation}
    \theta_{\oplus}(t) = \omega_{\oplus} \cdot t = \frac{2\pi}{86400} \cdot t
\end{equation}

\subsection{Transformation ECI vers ECEF}

La transformation du repère inertiel ECI vers le repère tournant ECEF s'effectue par une rotation autour de l'axe polaire :

\begin{equation}
    \begin{pmatrix}
        x_{\text{ECEF}} \\
        y_{\text{ECEF}} \\
        z_{\text{ECEF}}
    \end{pmatrix}
    =
    \begin{pmatrix}
        \cos\theta_{\oplus} & \sin\theta_{\oplus} & 0 \\
        -\sin\theta_{\oplus} & \cos\theta_{\oplus} & 0 \\
        0 & 0 & 1
    \end{pmatrix}
    \begin{pmatrix}
        x \\
        y \\
        z
    \end{pmatrix}
\end{equation}

\subsection{Coordonnées géographiques}

Depuis les coordonnées cartésiennes ECEF, on peut calculer latitude $\phi$ et longitude $\lambda$ :

\begin{align}
    \phi &= \arcsin\left(\frac{z_{\text{ECEF}}}{r}\right) \\
    \lambda &= \arctan\left(\frac{y_{\text{ECEF}}}{x_{\text{ECEF}}}\right)
\end{align}

\section{Liens inter-satellites}

\subsection{Distance entre deux satellites}

Soit deux satellites aux positions $\vec{r}_1 = (x_1, y_1, z_1)$ et $\vec{r}_2 = (x_2, y_2, z_2)$. La distance euclidienne est :

\begin{equation}
    d_{12} = \|\vec{r}_2 - \vec{r}_1\| = \sqrt{(x_2-x_1)^2 + (y_2-y_1)^2 + (z_2-z_1)^2}
\end{equation}

\subsection{Latence de communication}

La latence de propagation $\tau$ d'un signal électromagnétique entre les deux satellites est :

\begin{equation}
    \tau = \frac{d_{12}}{c}
\end{equation}

Pour les liens ISL (Inter-Satellite Links) en LEO, typiquement $d_{12} \leq 5000$ km, donnant $\tau \leq 16.7$ ms.

\subsection{Visibilité ligne de vue}

Deux satellites sont en ligne de vue si le segment les reliant n'intersecte pas la Terre.

Soit $\vec{r}_1$ et $\vec{r}_2$ les positions des satellites. Le segment est paramétré par :

\begin{equation}
    \vec{r}(u) = \vec{r}_1 + u(\vec{r}_2 - \vec{r}_1), \quad u \in [0,1]
\end{equation}

L'occultation terrestre se produit si :

\begin{equation}
    \exists u \in [0,1] : \|\vec{r}(u)\| < R_{\oplus}
\end{equation}

La distance minimale du segment au centre de la Terre est :

\begin{equation}
    d_{\min} = \frac{\|\vec{r}_1 \times \vec{r}_2\|}{\|\vec{r}_2 - \vec{r}_1\|}
\end{equation}

Les satellites sont en ligne de vue si et seulement si :

\begin{equation}
    d_{\min} > R_{\oplus}
\end{equation}

\section{Liens intra-plan et inter-plan}

\subsection{Liens intra-plan}

Dans un plan orbital $p$ contenant $N_p$ satellites, chaque satellite $s$ établit des liens avec ses voisins immédiats :

\begin{itemize}
    \item Satellite précédent : $(p, (s-1) \bmod N_p)$
    \item Satellite suivant : $(p, (s+1) \bmod N_p)$
\end{itemize}

La distance angulaire entre satellites voisins dans le plan est :

\begin{equation}
    \Delta\nu = \frac{2\pi}{N_p}
\end{equation}

La distance physique entre satellites voisins dans le plan varie avec leur position relative, mais reste approximativement :

\begin{equation}
    d_{\text{intra}} \approx 2r\sin\left(\frac{\pi}{N_p}\right)
\end{equation}

\subsection{Liens inter-plan}

Les liens inter-plan connectent les satellites de plans adjacents. Pour un satellite dans le plan $p$, les liens inter-plan potentiels sont avec les satellites du plan $(p+1) \bmod P$.

La distance entre deux plans adjacents au niveau de l'équateur est :

\begin{equation}
    d_{\text{équateur}} = 2r\sin\left(\frac{\pi}{P}\right)
\end{equation}

\section{Stabilité des liens}

\subsection{Condition de stabilité}

Une constellation est dite \textbf{stable} si tous les liens entre satellites voisins maintiennent une visibilité permanente durant toute la période orbitale, c'est-à-dire :

\begin{equation}
    \forall t \in [0, T_{\text{orb}}], \forall (i,j) \in \mathcal{L}_0 : d_{\min}^{ij}(t) > R_{\oplus}
\end{equation}

où $\mathcal{L}_0$ est l'ensemble des paires de satellites voisins à $t=0$.

\subsection{Variation de distance}

Bien que les liens restent établis, la distance entre satellites voisins varie au cours du temps. Pour un lien intra-plan, la variation est minimale car les satellites maintiennent leur espacement relatif.

Pour un lien inter-plan entre satellites $(p_1, s_1)$ et $(p_2, s_2)$, la distance varie selon :

\begin{equation}
    d_{12}(t) = \|\vec{r}_{p_1,s_1}(t) - \vec{r}_{p_2,s_2}(t)\|
\end{equation}

Cette variation induit des fluctuations de latence :

\begin{equation}
    \Delta\tau_{\max} = \frac{d_{\max} - d_{\min}}{c}
\end{equation}

\section{Couverture globale}

\subsection{Angle d'élévation minimum}

Pour qu'un satellite soit visible depuis un point au sol, l'angle d'élévation $\epsilon$ doit être supérieur à un minimum (typiquement $\epsilon_{\min} = 5°$ à $10°$).

L'angle d'élévation est donné par :

\begin{equation}
    \sin\epsilon = \frac{\|\vec{r}_{\text{sat}}\|^2 - R_{\oplus}^2 - d_{\text{sol-sat}}^2}{2R_{\oplus} d_{\text{sol-sat}}}
\end{equation}

où $d_{\text{sol-sat}}$ est la distance entre le point au sol et le satellite.

\subsection{Temps de visibilité}

Le temps pendant lequel un satellite reste visible depuis un point au sol (pour $\epsilon > \epsilon_{\min}$) dépend de l'altitude et de l'inclinaison :

\begin{equation}
    t_{\text{vis}} = \frac{2}{\omega} \arccos\left(\frac{R_{\oplus}\cos\epsilon_{\min}}{r}\right)
\end{equation}

\section{Exemple numérique}

Considérons une constellation Walker Delta $24/6/1$ à $h = 550$ km avec $i = 55°$.

\subsection{Paramètres calculés}

\begin{align}
    r &= 6371 + 550 = 6921 \text{ km} \\
    \omega &= \sqrt{\frac{398600.4418}{6921^3}} = 1.096 \times 10^{-3} \text{ rad/s} \\
    v &= \sqrt{\frac{398600.4418}{6921}} = 7.59 \text{ km/s} \\
    T_{\text{orb}} &= \frac{2\pi}{1.096 \times 10^{-3}} = 5731 \text{ s} = 95.5 \text{ min}
\end{align}

\subsection{Distribution des satellites}

Avec $T=24$ et $P=6$ :

\begin{align}
    N_p &= \left\lfloor \frac{24}{6} \right\rfloor = 4 \text{ satellites par plan} \\
    \Delta\Omega &= \frac{360°}{6} = 60° \\
    \Delta\nu_{\text{plan}} &= \frac{360°}{4} = 90°
\end{align}

RAAN des 6 plans : $\Omega \in \{0°, 60°, 120°, 180°, 240°, 300°\}$

\subsection{Phasage avec F=1}

Le décalage de phase entre plans adjacents est :

\begin{equation}
    \Delta\nu_{\text{phase}} = \frac{360° \times 1}{24} = 15°
\end{equation}

Pour le plan 0, satellite 0 : $\nu_{0,0} = 0° + 0° = 0°$

Pour le plan 1, satellite 0 : $\nu_{1,0} = 0° + 15° = 15°$

Pour le plan 2, satellite 0 : $\nu_{2,0} = 0° + 30° = 30°$

\section{Conclusion}

Ces équations permettent de modéliser complètement le mouvement d'une constellation Walker Delta :

\begin{itemize}
    \item Calcul des positions initiales selon le schéma de phasage
    \item Propagation temporelle des positions
    \item Calcul des distances et latences
    \item Vérification de la visibilité ligne de vue
    \item Analyse de la stabilité des liens
\end{itemize}

L'implémentation numérique de ces formules permet de simuler et d'optimiser les performances des constellations satellites pour des applications de communication et d'observation.

\end{document}
